\chapter{\'Etude des risques}
	Le pr\'esent projet est sujet \`a des risques qui pourraient emp\^echer sa r\'ealisation. Ces risques sont de deux groupes:

	\begin{itemize}
		\item[-] Les risques li\'ees \`a la construction de solution propos\'ee. Ce sont la d\'erive des objectifs, l'insuffisance de ressources et les incidents.
		\item[-] Les risques pour que malgr\'e le fait que la solution propos\'e soit construite et en production, le probl\`eme demeure entier. Ce sont la performance et la s\'ecurit\'e du syst\`eme.
	\end{itemize}

	\paragraph{} Dans ce chapitre, nous \'etudions ces risques et d\'ecidons comment les g\'erer.

	\section{Registre des risques}
		\subsubsection{D\'erive des objectifs\cite{DeriveDesObjectifs}}
			La d\'erive des objectifs est le fait d'ajouter de nouveaux livrables,  contraintes ou objectifs \`a un projet pendant son ex\'ecution. Elle peut \^etre engendr\'ee par l'\'evolution des exigences des principales parties prenantes du projet ou par une mauvaise communication entre les membres de l'\'equipe de d\'eveloppement. Ce n'est pas une mauvaise chose en soit, mais si elle n'est pas contr\^ol\'ee, la d\'erive des objectifs peut alourdir \'enorm\'ement le budget, causer une insuffisance de ressources ou diminuer la performance de l'\'equipe.

		\subsubsection{Ressources insuffisantes\cite{RessourcesInsuffisantes1, RessourcesInsuffisantes2}}
			Il s'agit du cas o\`u l'\'equipe ne r\'eussit pas ou plus \`a rassembler les ressources n\'ecessaires pour r\'ealiser des t\^aches faisant partie du projet. Ces ressources comprennent: le temps, les outils, les comp\'etences, les moyens financiers, etc. Selon sa gravit\'e, une insuffisance de ressources peut causer des retards dans la r\'ealisation d'un projet, ou provoquer son \'echec.

		\subsubsection{Incident}
			Dans le contexte d'ins\'ecurit\'e passager que se trouve le pays, toute personne qui y circule est sujet au risque de blessure potentiellement grave (voir m\^eme mortelle). D'autre part, cette situation cause de temps en temps des relogement forc\'es ou la d\'efaillance des r\'eseaux de communication. Si l'un de nous se trouvait victime de l'un de ces facteurs de risques, il pourrait se voir dans l'incapacit\'e de poursuivre le travail pendant un certain temps ou d\'efinitivement.


		\subsubsection{Faibles performances}
			La performance d'un site Web est la somme d'un ensemble de carat\'eristiques mesurables du site. Elle concerne essentiellement la vitesse de chargement des pages, la vitesse de r\'eponse aux clicks, la vitesse \`a laquelle l'utilisateur trouvera les informations recherch\'ees c'est-\`a-dire, l'ergonomie du site, la capacit\'e du site \`a susciter chez les visiteurs l'int\'er\^et d'aller au bout de leurs besoins et de revenir sur le site pour satisfaire des besoins similaires, sa position dans les r\'esultats de recherches sur les diff\'erents moteurs. Si le site n'est pas performant, les visiteurs le quitteront vite, n'y reviendront pas \`a l'avenir, garderont une mauvaise opinion de celui-ci, de plus il sera difficile de le trouver puisque mal class\'e dans les recherches.


		\subsubsection{S\'ecutit\'e}
			La s\'ecurit\'e d'un site web est une caract\'eristique de celui qui garanti que ses donn\'ees et services seront toujours disponible, que les donn\'ees qu'il fournit sont int\`egres, et que chaque utilisateur ne peut acc\`eder qu'\`a ce que ses privil\`eges lui permettent. Un site Web non-s\'ecuris\'e ou pas assez s\'ecuris\'e repr\'esente un danger \`a la fois pour son propr\'etaire que pour ces clients.


	\section{Plan de gestion des risques}
		\subsubsection{D\'erive des objectifs\cite{DeriveDesObjectifs}}
			 Il est n\'ecessaire de s'assurer qu'aucune modification dans le p\'erim\`etre d'un projet ne causera de probl\`eme. Pour cela, de telles modifications doivent \^etre soumis au processus de contr\^ole suivant:
				\begin{itemize}
					\item[-] Garder la version initiale du p\'erim\`etre du projet comme p\'erim\`etre de r\'ef\'erence.
					\item[-] Analyser l'\'ecart entre le p\'erim\`etre initial et le p\'erim\`etre apr\`es avoir effectu\'e une modification sp\'ecifique.
					\item[-] D\'eterminer la cause et le degr\'e des changements constat\'es.
					\item[-] Utiliser les r\'esultats de l'\'etape pr\'ec\'edent, les objectifs et les contraintes du projet pour d\'ecider si la modification peut effectivement avoir lieu.\\
				\end{itemize}

			\subsubsection{Ressources insuffisantes}
				Ce projet est financ\'e par le PNUD, nous garantissant une certaine stabilit\'e financi\`ere. De plus, nous faisons en sorte que nos d\'epenses soient optimales en assurant que nous disposions de ce qu'il faut avec une marge raisonnable pour g\'erer les impr\'evus. En ce qui concerne les ressources humaines et temporelle, nous conceptualisons une solution qui ne d\'epasse pas les comp\'etences et les capacit\'e d'apprentissage de l'\'equipe de d\'eveloppement et adoptons une m\'ethodologie efficace pour bien g\'erer le temps dont nous disposons.

			\subsubsection{Incident}
				Il est impossible de pr\'evoir ou d'emp\^echer ces \'ev\'enements. N\'eanmoins, nous voulons en att\'enuer les effets\cite{RessourcesInsuffisantes2}. Pour cela, nous nous pr\'eparons aux changements en ajoutant les points suivants \`a notre m\'ethodologie de travail.
				\begin{itemize}
					\item[-] Utiliser plusieurs m\'ethodes de communication diff\'erentes
					\item[-] Rapporter r\'eguli\`erement le statut de chaque travail \`a l'ensemble du groupe
					\item[-] Partager r\'eguli\`erement les derni\`eres travaux effectuer le reste du groupe
					\item[-] Documenter chaque travail r\'ealis\'e
					\item[-] Commenter les codes sources\\
				\end{itemize}

			\subsubsection{Faibles performances}
				Pour nous assurer de la performance du site, pendant sa construction, nous garderons la performance comme une contrainte qui influencera toutes les d\'ecisions que nous aurons \`a prendre et choix que nous aurons \`a faire. Ensuite, une fois le site d\'eploy\'e, pendant la p\'eriode de test, nous nous rassurons que le site est effectivement performant.

			\subsubsection{S\'ecutit\'e}
				Le Standard ASVS \footnote{Application Security Verification Standard} de OWASP \footnote{Open Worldwide Application Security Project} d\'efini un cadre fiable qui peut nous aider \`a d\'evelopper une application web s\'ecuris\'ee. 

			\subsubsection{Cas exceptionnels}
				Toutes situations que nous n'avons pas tra\^it\'e ici est consid\'er\'e comme un cas exceptionnel. A l'occurrence d'une telle situation, nous le rapporterons \`a notre client pour pouvoir la traiter conjointement avec lui.