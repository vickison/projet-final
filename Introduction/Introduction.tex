\chapter*{Mise en contexte}
    \addcontentsline{toc}{chapter}{Mise en contexte}
%--------------------------------------------------Introduction generale-------------------------------------------------------------------------

	\addcontentsline{toc}{section}{Introduction g\'en\'erale}

	\paragraph{} La R\'epublique d'Ha\"iti est un pays insulaire situ\'e dans les Cara\"ibes, partageant l'\^ile d'Ha\"iti avec la R\'epublique Dominicaine \`a l'est . Le pays a une superficie de 27 750 kilom\`etres carr\'es\cite{htsuperficie}, c'est le troisi\`eme pays le plus peupl\'e des Cara\"ibes, avec une population estim\'ee \`a pr\`es de 11,448 millions d'habitants.\cite{htpopulationA}\cite{htpopulationB} \\
	Ha\"iti se trouve dans une r\'egion g\'eographique sujette \`a des activit\'es sismiques et climatiques. Selon l'indice mondial de risque (WRI\footnote{World Risk Index \cite{WRI}}), Ha\"iti a l'un des indices de pr\'edisposition \footnote{Probabilit\'e qu'une soci\'et\'e ou qu'un \'ecosyst\`eme donn\'e soit endommag\'e en cas de catastrophe naturelle} les plus \'elev\'es au monde.\cite{RisqueEnHaiti}  \\
	Au cours de son histoire, Ha\"iti a malheureusement \'et\'e t\'emoin de s\'eismes d\'evastateurs, d'inondations, de glissement de terrain et d'autres catastrophes naturelles causant des pertes en vies humaines et des d\'eg\^ats mat\'eriels consid\'erables. Parmi eux nous pouvons citer le s\'eisme d\'evastateurs de 2010\cite{seisme2010} ou les ouragans Matthew \cite{ouraganMatthew1}\cite{ouraganMatthew2} et Jeanne \cite{ouraganJeanne} \\
	Outre les catastrophes naturelles, Ha\"iti fait \'egalement  face \`a des probl\`emes d'ins\'ecurit\'e sociale\cite{insecuriteSociale1} \cite{insecuriteSociale2}, d'ins\'ecurit\'e alimentaire\cite{insecuriteAlimentaire}, de sant\'e publique\cite{sante}, d'\'education\cite{education1}, et nous en passons.


	\paragraph{} La R\'epublique d'Ha\"iti a donc un niveau  de risque \'elev\'e et de nombreux d\'efis \`a relever. C'est dans le but d'adresser certains de ces prob\`emes que ce pr\'esent travail a vu le jour. Nous comptons bien contribuer \`a la r\'esolution graduelle de ces probl\`emes tout en fournissant \`a la communaut\'e un moyen fiable et efficace de se procurer des informations notamment sur la gestion des risques, mais aussi sur d'autres domaines importants.






%-----------Le problematique-----------------------
\section*{Probl\'ematique}%Ajouter de nouvelles source pour d\'efinir l'URG\'eo
\addcontentsline{toc}{section}{La probl\'ematique}
\label{SectionProblematique}

La situation de risques omnipr\'esentes auxquels Ha\"iti est expos\'e ne cesse de motiver diff\'erents acteurs de la communaut\'e \`a chercher des solutions potentielles \`a ce probl\`eme. C'est en ce sens que l'URG\'eo, L'Unit\'e de Recherche en G\'eosciences de la Facult\'e Des Sciences de l'Universit\'e d'\'Etat d'Ha\"iti, depuis sa cr\'eation en 2011, a d\'ej\`a r\'ealis\'e un grand nombre de travaux de recherche, gr\^ace auxquels il a produit et r\'ecolt\'e de nombreuses informations.  L'URG\'eo s'est servi \`a la fois de ses propres informations et d'autres acquises dans le cadre d'\'echanges avec des pairs  pour produire des documents num\'eriques illustr\'es sous différents formats: audios, vid\'eos, textuels et graphiques, en utilisant les avantages de l'illustration (Voir Section~\ref{SectionAvantageIllustrations} ) comme moyen de transmission de ces informations. \\


\noindent Aujourd'hui, l'URG\'eo dispose d'un ensemble d'illustrations (image, vid\'eo, texte et audio) concernant: \texttt{la pr\'evention des risques, l'\'equit\'e des genres, la sant\'e et les mines}. De plus, elle  pr\'evoit que ses travaux \`a venir ajouteront de nouveaux contenus \`a son stock,  non seulement sur les th\'ematiques pr\'ec\'edentes, mais aussi sur de nouvelles th\'ematiques.\par
\noindent L'URG\'eo souhaite que ces illustrations soient accessibles en ha\"iti et \`a travers le monde. \textbf{Quelle solution informatique permettrait d'effectuer cette t\^ache.}

\paragraph{}Ce projet fait partie d'un plus grand projet de l'URG\'eo souhaitant \'eduquer la population ha\"itienne sur les risques auxquels elle est expos\'ee, et voulant sensibiliser \`a la compr\'ehension de ces ph\'enom\`enes et \`a les g\'erer plus efficacement. \\
Le processus pour informer une communaut\'e peut \^etre divis\'e en trois grandes \'etapes: la collecte et la production des informations \`a travers la recherche et le d\'eveloppement, les d\'emarches visant \`a  les rendre accessibles, et enfin la promotion de ces informations. Bien entendu, comme nous venons de le voir, le pr\'esent projet s'int\'eresse \`a la deuxi\`eme \'etape \`a savoir l'accessibilit\'e des informations. \\
Les fichiers contenant les informations repr\'esentant les diff\'erentes illustrations sont produits et mises en forme dans le cadre d'activit\'es qui ne concernent pas le pr\'esent projet. Notre t\^ache consiste \`a concevoir une solution informatique pour rendre ces informations accessibles depuis le monde entier. L'URG\'eo se chargera d'utiliser cette solution pour exploiter son stock d'illustrations et ainsi atteindre son objectif.



%-------------Exigences---------------------------

	\section*{Besoins du client}
	\addcontentsline{toc}{section}{Les besoins}

		Les exigences du client sont les suivants:

		\subsubsection*{Besoins fonctionnels}
		\label{SectionBesoinsFonctionnels}
		\addcontentsline{toc}{subsection}{Besoins fonctionnels}
			\begin{itemize}
				\item[-] La solution doit permettre de stocker les illustrations
				\item[-] L'utilisateur doit pouvoir rechercher un ou plusieurs contenu(s) en particulier
				\item[-] La solution doit permettre aux utilisateurs de visionner les contenus
				\item[-] Le client doit pouvoir cr\'eer de nouvelles cat\'egories
				\item[-] L'utilisateur doit pouvoir t\'el\'echarger des contenus
			\end{itemize}

		\subsubsection*{Besoins non-fonctionnels}
		\addcontentsline{toc}{subsection}{Besoins non-fonctionnels}
			\begin{itemize}
				\item[-] Les contenus doivent \^etre accessibles via internet
				\item[-] Tous les outils utilis\'es doivent \^etre sous licence libre
				\item[-] Les contenus doivent \^etre organis\'es en Cat\'egories
				\item[-] L'interface utilisateur doit contenir le moins de texte possible
				\item[-] Les outils utilis\'es doivent \^etre opensource
				\item[-] La solution doit \^etre accompagn\'ee d'un document
				\item[-] Ce document doit contenir les d\'etails de la conception de la solution en question
				\item[-] Le document doit \^etre r\'edig\'e en Latex
			\end{itemize}



%----------------------------------
\section*{Plan et int\'er\^et du travail}
\addcontentsline{toc}{section}{Plan et int\'er\^et du travail}
	Nous allons, d'abord, faire l'\'etat des connaissances sur le sujet. Ensuite, nous pr\'esenterons la solution que nous proposons \`a la probl\'ematique. Apr\`es  cela, nous d\'etaillerons l'analyse, la conception, la construction et le d\'eploiement de la solution.\\
	Le travail qui en d\'ecoulera, un premier pas vers la valorisation de ces contenus, jouera un r\^ole essentiel dans l'exploitation des ressources de l'URG\'eo pour aboutir aux r\'esultats suivants:

	\begin{itemize}
		\item[-] Avant un d\'esastre, il permettra de rendre accessibles les connaissances sur les risques encourus en Ha\"iti et les moyens de les pr\'evenir.
		\item[-] Pendant un d\'esastre, il permettra d'acc\'eder aux connaissances ad\'equates pour emp\^echer, ou du moins, limiter les d\'eg\^ats.
		\item[-] Apr\'es un d\'esastre, il aidera \`a r\'eparer les d\'eg\^ats s'il y en a eu, et, si n\'ecessaire, \`a rendre accessibles les nouvelles connaissances acquises par l'exp\'erience sur le sujet.\\
	\end{itemize}

	\noindent Nous esp\'erons ainsi contribuer \`a la formation de citoyens instruits et engag\'es, capables de participer activement \`a la construction d'un environnement plus sain, en leur permettant de bien comprendre certains de leurs probl\`emes, d'apprendre \`a les apr\'ehender, \`a se prot\'eger, et \`a les surmonter.


