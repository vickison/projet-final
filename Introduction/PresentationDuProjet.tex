\chapter{Pr\'esentation du projet}


	\section{P\'erim\`etre du projet \cite{PerimetreDUnProjet}}
		 \subsubsection{Solution propos\'ee}
		 	Apr\`es avoir analys\'e les besoins du client et avoir fait l'\'etat des connaissances sur les domaines autour desquels se d\'eroule le projet, nous proposons, comme solution \`a la probl\'ematique, de construire une e-librairie gr\^ace \`a laquelle les illustrations seront distribu\'ees sous la forme de documents.

		\subsubsection{Entr\'ees du syst\`eme}
			Le syst\`eme re\c{c}ois en entr\'ees principales les requ\^etes des utilisateurs.

		\subsubsection{Sorties de syst\`eme}
			Le syst\`eme permet de:
			\begin{itemize}
				\item[-] Visualiser les documents
				\item[-] Filtrer les documents suivant leur cat\'egorie
				\item[-] Trier les documents sur le site
				\item[-] Noter un document
				\item[-] Rechercher document
				\item[-] Ajouter/Supprimer/Modifier un document
				\item[-] Ajouter/Supprimer/Modifier une cat\'egorie
				\item[-] Ajouter/Supprimer/Modifier un administrateur
			\end{itemize}

		\subsubsection{Les livrables}
			Une fois le d\'eveloppement termin\'e, nous fournirons \`a notre client:
				\begin{itemize}
					\item[-] Une application web en production o\`u il pourra r\'eclamer les services fournies par la librairie.
					\item[-] Une documentation technique qui retrace les \'etapes de la construction du site, d\'ecrit son fonctionnement et qui explique comment utiliser son interface.
				\end{itemize}


		\subsubsection{Facteurs de qualit\'e du syst\`eme}
		\label{SectQualiteDuSysteme}
			\paragraph{} En plus de satisfaire aux besoins exprim\'es par le client, le projet doit satisfaire certains crit\`eres, dit de qualit\'es, garantissant sa capacit\'e \`a effectuer la t\^ache pour laquelle elle a \'et\'e con\c{c}u de fa\c{c}on efficace et efficiente \`a la fois dans son environnement de d\'eploiement et dans son contexte de production.\cite{FacteursDeQualiteDUnSysteme}.
			\\
			Les principaux facteurs qui garantirons la qualit\'e du projet sont les suivants:
			\begin{itemize}
				\item[-] Les services qu'il (le syst\`eme) peut fournir doivent \^etre pertinents pour bien r\'epondre aux besoins du clients ainsi qu'aux besoins et attentes des utilisateurs.
				\item[-] L'ergonomie du site doit offrir une bonne exp\'erience aux utilisateurs.
				\item[-] L'interface utilisateur doit \^etre intuitif et facile \`a comprendre et \`a utiliser quels que soient le dispositif ou le navigateur des utilisateurs
				\item[-] Le d\'elai de chargement des ressources doit \^etre convenable (pas excessif).
				\item[-] Le syst\`eme doit respecter les normes de s\'ecurit\'e.
				\item[-] Le site doit \^etre adapt\'e \`a toutes les tailles d'\'ecrans d'appareils.
				\item[-] Le site doit \^etre facilement trouv\'e par les moteurs de recherche.
			\end{itemize}

		\subsubsection{Contraintes}
			\paragraph{} Les contraintes du projet sont les suivants:
				\begin{itemize}
					\item[-] Le d\'elai de livraison est de 6 mois allant du 1\textsuperscript{er} Avril 2023 au 1\textsuperscript{er} Octobre 2023.
				\end{itemize}

		\subsubsection{Exclusion du projet}
			\paragraph{} Le projet n'inclut pas les \'el\'ements suivants:
				\begin{itemize}
					\item[-] La promotion et le r\'ef\'erencement du site

				\end{itemize}



	\section{Apport du projet}
		\paragraph{}
			Il existe de nombreux librairies en ligne. Et notre projet est une parmi tant d'autres. Toutefois, son essence repose dans le but pour lequel il voit le jour, \`a savoir, \textit{fournir un moyen de s'informer sur les risques en Ha\"iti gr\^ace \`a des contenus illustr\'es}.

		\paragraph{} Les points qui rendent le projet unique, sont les suivants:\\
			\begin{itemize}
				\item[-] \projectName\ offre une plateforme unique sur laquelle on peut trouver les informations d\'esir\'ees. Cela permet d'\'economiser le temps et de diminuer l'effort qu'il faudrait fournir pour trouver ces informations en ligne.\\

				\item[-] \projectName\ fournit l'acc\`es \`a des contenus qui r\'efl\`etent l'environnement social ha\"itien dans les d\'etails et offre ainsi des solutions plus \`a la port\'ee de la communaut\'e locale.\\
				
				\item[-] \projectName\ utilise des illustrations pour communiquer ses informations, permettant ainsi de profiter des immenses atouts de l'illustration comme vecteur d'information (Voir Section~\ref{SectionAvantageIllustrations} - Importance de l'illustration)
			\end{itemize}